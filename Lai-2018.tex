\documentclass{beamer}
\usepackage[no-math]{fontspec}
\usepackage{xeCJK}
\setCJKmainfont{Source Han Sans TW}
\hypersetup{colorlinks,linkcolor=}

\usetheme{CambridgeUS}
\title[(Lai \textit{et al}, 2018)]{
    Robotic nipple-sparing mastectomy and immediate breast reconstruction with
    gel implant: technique, preliminary results and patient-reported cosmetic
    outcome
}
\subtitle{Hung-Wen Lai \textit{et al}, \textit{Ann Surg Oncol}, 2018}
\author[Chen-Pang He]{何震邦 (Chen-Pang He), Intern}
\date{March 5, 2019}
\institute[CGH]{Cathay General Hospital}

\newcommand*{\solo}[1]{\centering\includegraphics[width=\textwidth, height=0.8\textheight, keepaspectratio]{#1}}

\begin{document}
\maketitle

\begin{frame}{Endoscopic-assisted nipple-sparing mastectomy (E-NSM)}
    \begin{itemize}
        \item E-NSM was reported to be associated with small inconspicuous
              incisions and good cosmetic outcome.
        \item Conventional E-NSM was performed with two separate incisions over
              the axilla and peri-areolar regions.
        \item New technique modifications of E-NSM are emerging, focusing on
              single-axillary incision NSM, which spares the peri-areolar
              incision and thereby decreases the compromise of blood supply
              from the mastectomy skin flap, and was reported to have a low
              nipple areolar complex (NAC) necrosis rate (0\%).
    \end{itemize}
\end{frame}

\begin{frame}{Limitations of E-NSM}
    \begin{itemize}
        \item The 2D endoscopic in-line camera produces an inconsistent optical
              window around the curvature of the breast skin flap.
        \item Internal mobility was limited and the dissection angles were
              inadequate, with traditional endoscopic rigid tip instruments
              through a single access.
        \item Due to these limitations and difficulty, neither conventional nor
              single-access E-NSMs are widely used.
    \end{itemize}
\end{frame}

\begin{frame}{Robotic nipple-sparing mastectomy (R-NSM)}
    \begin{itemize}
        \item Robotic surgery incorporates a 3D imaging system, as well as
              flexibility of the arms and instruments.
        \item R-NSM was reported to have the potential to overcome the
              technique difficulty of E-NSM, and showed promising cosmetic
              outcome.
        \item On R-NSM and immediate breast reconstruction (IBR) with gel
              implant procedure in breast cancer patients was reported in this
              study.
        \item The technique, perioperative morbidity, preliminary oncologic
              safety, and patient-reported cosmetic outcomes were analyzed and
              reported.
    \end{itemize}
\end{frame}

\section{Methods}
\subsection{Patients}
\begin{frame}{Patients}
    \begin{itemize}
        \item Patients who received robotic breast surgeries from March 2017 to
              February 2018 at Changhua Christian Hospital
        \item A total of 31 robotic breast surgery procedures
            \begin{itemize}
                \item 28 female patients with breast cancer
                    \begin{itemize}
                        \item 3 with bilateral disease
                    \end{itemize}
                \item 29 R-NSM procedures
                    \begin{itemize}
                        \item 2 patients received bilateral R-NSM without reconstruction.
                        \item 2 patients received R-NSM and IBR with robotic-assisted harvesting of the latissimus dorsi flap.
                        \item 22 patients received 23 R-NSM and IBR with gel implant.
                    \end{itemize}
            \end{itemize}
    \end{itemize}
\end{frame}

\begin{frame}{Indications}
    \begin{itemize}
        \item Early-stage breast cancer ($\le$ IIIA)
        \item Tumor size < 5 cm
        \item No evidence of multiple lymph node metastasis
        \item No evidence of nipple, skin or chest wall invasion
        \item Non-inflammatory breast cancer
        \item No severe comorbidity
            \begin{itemize}
                \item Heart disease, renal failure, liver dysfunction, poor performance status, etc.
            \end{itemize}
        \item Women with large breasts\footnote{Specimen > 600 g or > E cup}
              and breast ptosis are not good candidates due to the difficulty
              of the technique and suboptimal cosmetic outcome.
    \end{itemize}
\end{frame}

\subsection{Techniques}
\begin{frame}
    \solo{F1a.png}
\end{frame}

\begin{frame}
    \solo{F1d.png}
\end{frame}

\begin{frame}
    \solo{F1f.png}
\end{frame}

\begin{frame}
    \solo{F1h.png}
\end{frame}

\begin{frame}{Video of the procedure}
    \centering\url{https://doi.org/10.1245/s10434-018-6704-2}
\end{frame}

\begin{frame}
    \solo{F2a.jpg}
\end{frame}

\begin{frame}
    \solo{F2d.jpg}
\end{frame}

\section{Results}
\begin{frame}{Pre- and postop photos}
    \solo{F3a.jpg}
    \footnotetext[1]{49 y/o, DCIS}
    \footnotetext[2]{59 y/o, metastatis in frozen biopsy}
\end{frame}

\begin{frame}
    \solo{F3g.eps}
\end{frame}

\begin{frame}
    \solo{T1a.eps}
\end{frame}

\begin{frame}
    \solo{T1b.eps}
\end{frame}

\begin{frame}
    \solo{T1c.eps}
\end{frame}

\begin{frame}
    \solo{T1d.eps}
\end{frame}

\begin{frame}
    \solo{T1e.eps}
\end{frame}

\begin{frame}
    \solo{T1f.eps}
\end{frame}

\begin{frame}
    \solo{T2a.eps}
\end{frame}

\begin{frame}
    \solo{T2b.eps}
\end{frame}

\begin{frame}
    \solo{T3.eps}
\end{frame}

\end{document}
